\documentclass[10pt,a4paper]{article}
\usepackage[utf8]{inputenc}
\usepackage[letterpaper,top=1cm,bottom=1cm,left=2cm,right=2cm,marginparwidth=1.75cm]{geometry}
\usepackage{graphicx}
\usepackage{subfig}
\usepackage{caption}
\usepackage{subcaption}
\title{Matplotlib}
\author{Aryan Vora (200204)}
\date{ 17-06-2021}

\begin{document}

\maketitle

\section{An Introduction to Matplotlib}
Matplotlib is a library in Python that offers functions to plot data. More specifically, we use the matplotlib.pyplot library.
\subsection{Importing matplotlib.pyplot}
This is how we import the library into the Python environment: Figure
\ref{fig:import1}
\begin{figure}[!h]
\centering
\includegraphics[width=1\textwidth]{import.png}
\caption{\label{fig:import1}Importing matlplotlib.pyplot.}
\end{figure}
\subsection{The plot() function}
The library offers a plot() function that can be used to plot 2 sets of data against each other. The syntax for the plot() function is as follows:
    plt.plot(x\_data,y\_data)
where x\_data and y\_data are the sets(may be numpy arrays, pandas dataframes or lists) for the data along x and y axes respectively.
There are further attributes to the plot() function, a few of which include:
\begin{itemize}
\item Labels of the axes
\item Color of the plot
\item Type of line used (solid, dashed, etc...)
\item Title of the plot
\item Legends
\end{itemize}
. However, note that plot() is used to plot a continuous graph. To plot discrete points, matplotlib.pyplot offers the scatter() function which is very similar to the plot() function in the way that it takes parameters as well as attributes.
The plot() function also offers many types of plots such as Mollweide projections, a very common projection used in astronomy as well as cartography.An implementation of the Mollweide projection is attached in Figure \ref{fig:example}.
\begin{figure}[!h]
    \centering
    \subfloat[Code behind the plot]{{\includegraphics[width=0.45\textwidth]{code.png} }}
    \qquad
    \subfloat[The plot]{{\includegraphics[width=0.45\textwidth]{star.png} }}
    \caption{A Mollweide porjection classifying LMXB (low mass X-ray binaries) and HMXB (high mass X-ray binaries) based on whether or not they were detected by AstroSAT}
    \label{fig:example}
\end{figure}
\subsection{Subplots}
The subplot() utility of matplotplib allows us to create multiple subplots in a single figure. An example of subplots showing LMXB and HMXB is attached in Figure \ref{fig:sub}. Note the distinct arc in the subplot for HMXB sources, this is the Milky Galaxy observed in the Mollweide Projection.
\begin{figure}[!h]
\centering
\includegraphics[width=1.0\textwidth]{subplot.png}
\caption{\label{fig:sub}HMXB and LMXB sources.}
\end{figure}
\subsection{Applications of matplotlib in astronomy}
We can use these techniques to plot constellations as well as plot and analyze data gathered by several sources.

Attached in Figure \ref{fig:strain1} is a plot of strain data versus time for a black-hole binary.
\begin{figure}[!ht]
\centering
\includegraphics[width=0.4\textwidth]{strain_time.png}
\caption{\label{fig:strain1}Plot of Strain data vs Time.}
\end{figure}
Figure \ref{fig:orion} shows the constellation Orion plotted in a stereographic projection using matplotlib. The size of each star is determined using its measured intensity (stellar magnitude).
\begin{figure}[!ht]
\centering
\includegraphics[width=0.4\textwidth]{orion.png}
\caption{\label{fig:orion}Orion.}
\end{figure}


\end{document}
